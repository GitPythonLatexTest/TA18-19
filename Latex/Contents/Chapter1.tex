!! read the instructions carefully !!

Here's the file for table, please note that a brief explanation is required

Please write your name, npm, and your current title project (new section)



\section{M. Amran Hakim Siregar / 1154106 / D4TI4B}
\begin{enumerate}
    \item Phase Heuristic Miner :
    \begin{equation}
        f=1/2=\bigg(1-\frac{\sum_{i}^{k}= 1 n_{i}m_{1}}{\sum_{i}^{k} = 1 n_{i}c_{i}}\bigg+ 1/2\big(\frac{\sum_{i}^{k}=1 n_{i}r_{i}}{\sum_{i}^{k}= 1 n_{i}p_{i}}\bigg)
    \end{equation}
\end{enumerate}
\par
The formula above is a formula in the phase heuristic miner after performing the event log stage to determine the data to be used.

\section{Faisal Syarifuddin / 1154104 / D4TI4B}

\begin{center}
    \begin{table}[hhhh]
    \caption{Phase Enhacement}
        \centering
        \begin{tabular}{cccc}
        \hline No & Years & Activity & Processing Time\\
        \hline
            A & 2025 & Dating & One time working  \\
            B & 2024 & Need Vendor & One time working  \\
            C & 2023 & Lose Vendor & One time working  \\
            ... & ... & ... & ...  \\
            O & 2018 & Vendor & One time working  \\
        \hline
        \end{tabular}
    \end{table}
\end{center}

\par In table \ref{table} there are 79 items retrieved from the data company. the data is a list of items that will be in use in project LPG, which the client PT. XXX.

\section{Maulyanda 1154008 D4TI3A}
\begin{table}[h]
    \centering
    \begin{tabular}{ccc}
    \hline
        Actual & Dloc & LES  \\
    \hline
         Dloc & TD & FD \\
         LES & TL & FL \\
    \hline
    \end{tabular}
    \caption{Calculation Result}
    \label{table1}
\end{table}

\par
in the \ref{table1} table there are 100 locations of items from the company PT. Y and PT. X

\section{cahya kurniawan 1154038 d4ti4d}
in table \ref{tabel1} describes the results of data acquisition that has been done using formula \ref{eq:1}, formula \ref{eq:2}, and formula \ref{eq:3}.

\begin{table}[h]
    \centering
    \begin{tabular}{|c|c|c|c|c|}
    \hline
    \multicolumn{3}{|c|}{distance between cluster centers} & Sub-district & amount of data\\
    \hline
    C1 & C2 & 3,16227766 & PORONG & 2849\\
    C1 & C3 & 5,830951895 & SUKODONO & 133\\
    C2 & C3 & 2,828427125 & BALONGBENDO & 133\\
    \hline
    & & BCV & 11,82165668&  \\
    \hline
    \multicolumn{4}{|c|}{Ratio} & 0\\
    \hline
    \end{tabular}
    \caption{THE RESULT DATA}
    \label{tabel1}
\end{table}
\end{partbacktext}


\section{Muhammad Nur Ikhsan 1154087 D4TI3D}
\begin{table}[h]
\caption{Primary Data}
\centering
\begin{tabular}{ccc}
\hline
Case Number & Part Number & First Location \\
\hline
PDAID1838D001X0059 & 12356987 & ROW H \\
PDAID1838D001X0065 & 55879226 & ROW J \\
PDAID1838D001X0058 & 66987822 & ROW K \\
\hline
\end{tabular}
\label{table2}
\end{table}

/section{Achmad Fatahillah 1154004 D4TI4A}
\begin{table}[h]
\caption{Result Sample}
\centering
\begin{tabular}{ccc}
\hline
Cluster & Max Temperature & Min \\
\hline
0 & 30 & 18
1 & 30 & 17
\hline
\end{tabular}
\label{temperature}
\end{table}

\section{Luqman Nurfajri 1154054 D4TI4A}
\begin{table}[h]
\caption{August Research Sample}
\centering
\begin{tabular}{ccc}
\hline
Date & Day & Temperature \\
\hline
1 & Wednesday & 23..1 \\
2 & Thursday & 21.9 \\
\hline
\end{tabular}
\label{table2}
\end{table}


