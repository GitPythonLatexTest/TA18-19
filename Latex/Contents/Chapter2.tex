!! read the instructions carefully !!

Here's the file for formula, please not that a brief explanation is required

Please write your name, npm, and your current title project(new section)



\section{Faisal Syarifuddin 1154104 D4TI4B}



\section{Value of Fitness}
    \label{fitness}
    \begin{equation}
        \frac{fo1 + f02 + f03 + ... + foN}{total fo}
    \end{equation}
    \par Description of \ref{fitness}:
    \par fo = function objective

\section{Find the probability}
    \label{ptobabilty}
    \begin{equation}
        P[i] = \frac{fitness[i]}{total_fitness}
    \end{equation}
    \par Description of \ref{ptobabilty}:
    \par P[i] = probability
    \par fitness[i] = value of fitness
    \par total fitness = sum which from value of fitness

\section{Maulyanda 1154008 D4TI3A}
\subsection{RANSAC Algorithm}
    \label{Formula}
    \begin{equation}
        k = \frac{log(1-p)}{log(1-W^n}
    \end{equation}

    the formula \ref{formula} RANSAC Algorithm used to calculate the accuracy of the location goods

\section{cahya kurniawan 1154038 d4ti4d}
\par
\begin{equation}
\label{eq:1}
\sqrt{(1-2)\sp{2} + (1-4)\sp{2}}
    \end{equation}
\par
The distance formula between cluster centers c1 with c3
\begin{equation}
\label{eq:2}
\sqrt{(1-4)\sp{2} + (1-6)\sp{2}}
    \end{equation}
\par
The distance formula between cluster centers c2 with c3
\begin{equation}
\label{eq:3}
\sqrt{(2-4)\sp{2} + (4-6)\sp{2}}
    \end{equation}


\section{Muhamad Nur Ikhsan 1154087 D4TI3D}
\par Destination Function
\begin{equation}
  \sum \limits_{i=1}^{n} {P_i} {x_i}
  \label{eq:4}
\end{equation}

\par Barrier function
\begin{equation}
  \sum \limits_{i=1}^{n} {W_i} {x_i} \leq M
  \label{eq:5}
\end{equation}


\section{Luqman Nurfajri 1154054 D4TI4A}
\begin{table} [h]
\caption{August Research Sample}
\Centering
\begin{tabular]{lcr}
\hline
Date & Day & Air Temperature \\
\hline
1 & Wednesday & 23.1   \\
2 & Thursday & 21.9 \\
\end{tabular}
\end{tabel}
